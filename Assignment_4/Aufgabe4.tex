\section{Angriffe}


\subsection*{Eavsdropping Attack}

Mit folgendem Befehl wurden die Messdaten konkateniert (ohne LF):
\begin{python}
python konk_data.py -A 1_ohneBwg_A_B.csv 2_mBwg_A_B.csv 3.a_mBwg_A_B.csv 
3.a_ohneBwg_A_B.csv 3.b_mBwg_A_B.csv 3.b_ohneBwg_A_B.csv 4_mBwg_A_B.csv 
5_mBwg_A_B.csv -B 1_ohneBwg_B_A.csv 2_mBwg_B_A.csv 3.a_mBwg_B_A.csv 
3.a_ohneBwg_B_A.csv 3.b_mBwg_B_A.csv 3.b_ohneBwg_B_A.csv 4_mBwg_B_A.csv 
5_mBwg_B_A.csv -E 1_ohneBwg_E_A.csv 2_mBwg_E_A.csv 3.a_mBwg_E_A.csv
3.a_ohneBwg_E_A.csv 3.b_mBwg_E_A.csv 3.b_ohneBwg_E_A.csv 4_mBwg_E_A.csv
5_mBwg_E_A.csv
\end{python}

Die Ausgaben namens \textit{A.csv}, \textit{B.csv} und 
\textit{E.csv} können
\href{https://mega.nz/file/Kt5xBZAB#WRGFcohI7XJrf5DPk1q67PtrjdRBoHcYXyj7-F8Ai3g}
{hier} runtergeladen werden.\\~\\

Folgende Befehle wurden zum Testen des Frameworks benutzt.


\begin{python}
python physec_praktikum.py -A A.csv -B B.csv -E E.csv -X 2 -Q 0
python physec_praktikum.py -A A.csv -B B.csv -E E.csv -X 2 -Q 1
\end{python}

Die Ausgaben des Frameworks können 
\href{https://mega.nz/file/iooRSRaT#bSVf5SFQHdotoiB_lNudiWK0QEkD0q-4O-qF6iZUipo}
{hier}
runtergeladen werden.\\



\subsubsection*{Quantisierer im direkten Vergleich}
Schaut man sich Abbildungen~\ref{fig:4.1} und ~\ref{fig:4.2} an, 
so erkennt man merkbare Unterschiede zwischen beiden 
\textit{Quantisierern}.    

\begin{figure}[hbt!]
	\centering
		\includegraphics[width=0.65\textwidth ]
		{Bilder/a4-q0-2.png}
		\caption{\textit{quant0}}
		\label{fig:4.1}
\end{figure}


\begin{figure}[hbt!]
	\centering
		\includegraphics[width=0.65\textwidth ]
		{Bilder/a4-q1-2.png}
		\caption{\textit{quant1}}
		\label{fig:4.2}
\end{figure}


\begin{figure}[hbt!]
	\centering
		\includegraphics[width=0.65\textwidth ]
		{Bilder/a4-q0-1.png}
		\caption{\textit{Signalstärkenverlauf}}
		\label{fig:4.3}
\end{figure}

\begin{comment}
Wenn man Abbildung~\ref{fig:4.1} betrachtet, so erkennt man, 
dass Einflüsse wie Bewegung und Distanz eine große Rolle 
spielen.\\
Analysiert man die Unterschiede so wird klar, dass Bewegung
und größere Distanz einen Angreifer benachteiligen.	
\end{comment}

\newpage
Als nächstes wurde mit den Ausgaben dieses Befehls getestet.
\begin{python}
python konk_data.py -A 1_ohneBwg_A_B.csv 2_mBwg_A_B.csv -B 1_ohneBwg_B_A.csv 
2_mBwg_B_A.csv -E 1_ohneBwg_E_A.csv 2_mBwg_E_A.csv
\end{python}


\begin{figure}[hbt!]
	\centering
		\includegraphics[width=0.65\textwidth ]
		{Bilder/a4-q0-1bis2.png}
		\caption{\textit{Signalstärkenverlauf}}
		\label{fig:4.4}
\end{figure}


\begin{figure}[hbt!]
	\centering
		\includegraphics[width=0.65\textwidth ]
		{Bilder/a4-q0-1bis2-1.png}
		\caption{\textit{quant0}}
		\label{fig:4.5}
\end{figure}
\clearpage

\begin{figure}[hbt!]
	\centering
		\includegraphics[width=0.65\textwidth ]
		{Bilder/a4-q1-1bis2-1.png}
		\caption{\textit{quant1}}
		\label{fig:4.6}
\end{figure}


Anschließend wurde mit den Ausgaben dieses Befehls getestet.
\begin{python}
python konk_data.py -A 3.a_mBwg_A_B.csv 3.a_ohneBwg_A_B.csv 3.b_mBwg_A_B.csv 
3.b_ohneBwg_A_B.csv  -B 3.a_mBwg_B_A.csv 3.a_ohneBwg_B_A.csv 3.b_mBwg_B_A.csv 
3.b_ohneBwg_B_A.csv -E 3.a_mBwg_E_A.csv 3.a_ohneBwg_E_A.csv 3.b_mBwg_E_A.csv 
3.b_ohneBwg_E_A.csv 
\end{python}

\begin{figure}[hbt!]
	\centering
		\includegraphics[width=0.65\textwidth ]
		{Bilder/a4-q0-3.png}
		\caption{\textit{Signalstärkenverlauf}}
		\label{fig:4.7}
\end{figure}


\begin{figure}[hbt!]
	\centering
		\includegraphics[width=0.65\textwidth ]
		{Bilder/a4-q0-3-1.png}
		\caption{\textit{quant0}}
		\label{fig:4.8}
\end{figure}


\begin{figure}[hbt!]
	\centering
		\includegraphics[width=0.65\textwidth ]
		{Bilder/a4-q1-3-1.png}
		\caption{\textit{quant1}}
		\label{fig:4.9}
\end{figure}

\subsubsection*{Resümee: Eavsdropping Attack}

\glqq Gibt es einen Zusammenhang zwischen dem Verwendeten
Quantisierer und der Stärke des Angreifers?\grqq\\

Wie Abbildungen~\ref{fig:4.4} und ~\ref{fig:4.5} zeigen, 
bietet der  \textit{quant0} beziehungsweise 
\textit{mean quantizer} dem Angreifer einen Vorteil.\\

\glqq Welche Rolle spielt die Entfernung des Angreifers 
zu Alice/Bob\grqq\\

In Abbildungen~\ref{fig:4.8} und ~\ref{fig:4.9} ist klar zu 
erkennen, dass zunehmende Distanz einen Angreifer 
benachteiligt.\\

\glqq Welche Rolle spielt die Bewegung im Raum?\grqq\\

Bewegung im Raum benachteiligt den Angreifer. Die 
Wirkung davon sieht man in Abbildungen~\ref{fig:4.4} 
und~\ref{fig:4.5}.


\subsection*{Repetition Attack}
Wie in der Vorlesung erwähnt wurde, ist es möglich
von einem Angreifer zyklische Bewegungen
auszunutzen, siehe Beispiel Modelleisenbahn.

\begin{figure}[hbt!]
	\centering
		\includegraphics[width=1\textwidth ]
		{Bilder/a4-2.png}
		\caption{\textit{Signalstärkenverlauf}}
		\label{fig:4.10}
\end{figure}


\begin{figure}[hbt!]
	\centering
		\includegraphics[width=0.7\textwidth ]
		{Bilder/a4-2-q0.png}
		\caption{\textit{quant0}}
		\label{fig:4.11}
\end{figure}


\begin{figure}[hbt!]
	\centering
		\includegraphics[width=0.7\textwidth ]
		{Bilder/a4-2-q1.png}
		\caption{\textit{quant1}}
		\label{fig:4.12}
\end{figure}
\clearpage

Es sind Muster zu erkennen, wobei 
Eves Muster eigentlich noch gleichmäßiger wäre,
aber die Messbedingungen waren nicht ideal.\\[3mm]
Es gibt erkennbare Unterschiede zwischen den
Quantisierern, wobei Eve \textit{quant0} bevorzugen
würde. Zyklische Bewegungen scheinen tatsächlich 
ausnutzbar zu sein für einen Angreifer. Eve muss dabei
unter anderem die Periode richtig schätzen.


\subsection*{Prediction Attack}

\begin{figure}[hbt!]
	\centering
		\includegraphics[width=1\textwidth ]
		{Bilder/a4-3.png}
		\caption{\textit{Signalstärkenverlauf}}
		\label{fig:4.13}
\end{figure}

Betrachtet man Abbildungen~\ref{fig:4.14} und 
~\ref{fig:4.15}
kann man zum Entschluss kommen, dass Eve zumindest 
bei \textit{quant0} diese Attacke nutzen könnte um die 
Entropie zwischen Alice und Bob möglichst weit zu 
reduzieren. Alternativ könnte man diesen Angriff als 
\textit{denial-of-service-Attacke} nutzen um eine
erfolgreiche Schlüsseleinigung zwischen Alice und Bob 
zu torpedieren oder sogar zu verhindern.

\begin{figure}[hbt!]
	\centering
		\includegraphics[width=0.7\textwidth ]
		{Bilder/a4-3-q0.png}
		\caption{\textit{quant0}}
		\label{fig:4.14}
\end{figure}


\begin{figure}[hbt!]
	\centering
		\includegraphics[width=0.7\textwidth ]
		{Bilder/a4-3-q1.png}
		\caption{\textit{quant1}}
		\label{fig:4.15}
\end{figure}

\clearpage

\subsection*{Eigener Angriff}
Text folgt

\begin{comment}
\begin{figure}[hbt!]
	\centering
		\includegraphics[width=0.7\textwidth ]
		{Bilder/a4-4.png}
		\caption{\textit{Vierfach gefaltete Alufolie}}
		\label{fig:4.1x}
\end{figure}	
\end{comment}
