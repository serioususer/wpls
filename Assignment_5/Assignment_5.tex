\documentclass[12pt,a4paper]{article}
\usepackage[german]{babel}
\usepackage[T1]{fontenc}
\usepackage[utf8x]{inputenc}
\usepackage{url}
\usepackage{graphicx}
\usepackage{algpseudocode}
\usepackage{algorithm}
\usepackage{geometry}
\usepackage{amsfonts}
\usepackage{amsmath}
\usepackage{tabularx}
\usepackage{txfonts} %Times New Roman Font
\usepackage{titlesec} %Format der Headings ändern
\usepackage{hyperref}
\usepackage{comment}
\usepackage{listings}
\usepackage{pythonhighlight}

\renewcommand{\thesection}{\arabic{section}.} %Nummerierung der Sections anpassen
\renewcommand{\labelenumi}{\alph{enumi})}  %Nummerierung der Listen anpassen
%\titleformat{\section}{\large\bfseries}{\thesection}{0.5em}{} %Format der Section Überschrift ändern
\setlength{\parindent}{0pt} %Keine Einrückung bei neuen Paragraphen
\geometry{left=2.0cm,textwidth=17cm,top=2.5cm,textheight=23cm}

% Anpassen %
%%%%%%%%%%%%%%%%%%%%%%%%%%%%%%%%%%%%%
\newcommand{\student}{Daniel Pantjuskin-Moos\\ 108013248222 } % Namen eintragen
\newcommand{\partner}{Vincent König\\ 108011232630} % Matrikelnummer eintragen
\newcommand{\group}{D} % Gruppennummer eintragen
%%%%%%%%%%%%%%%%%%%%%%%%%%%%%%%%%%%%%

\newcommand{\hwheadtwo}{$ $
  \vspace{-2cm}
  
\noindent \student \qquad \qquad  Wireless Physical Layer Security Praktikum \hfill SS 2020 \\
\noindent \partner \\
%\noindent \thirdone \\  % einkommentieren, falls ihr eine 3er Gruppe seid
\noindent Gruppe:~\group\\
$ $

  
\begin{center}    
{\Large \bf Abgabe PHYSEC 5}
\end{center}
}

\begin{document}
\hwheadtwo

\renewcommand{\thesection}{\arabic{section}}
\section{Bit Error Rate}


Hier Aufgabe 1 bearbeiten.\\
Nichts muss importiert werden.\\
Einfach starten wie gewohnt

\begin{figure}[hbt!]
	\centering
		\includegraphics[width=1\textwidth ]
		{Bilder/Testbild.jpg}
		\caption{Testbild kann gelöscht werden sobald gesehen}
		\label{fig:Labelx}
\end{figure}
\newpage
\subsection{Analyse I}


\begin{itemize}
    \item Vergleichen Sie das obige Messsetup mit dem von 
    Ihnen verwendeten Setup während des
    Praktikums und setzen Sie die mögliche Menge an 
    extrahierbaren Bits in Vergleich zu den
    RSSI-Werten
\end{itemize}

Da im jetzigen Assignment 5 mit CSI-Daten gearbeitet wird,
ist die zu verarbeitende Menge an Daten wesentlich größer.

\begin{itemize}
    \item Überlegen Sie, welchen Vorteil RSSI-Werte 
    gegenüber den komplexen Kanalimpulsantworten
    haben
\end{itemize}

RSSI-Werte sind einfacher zu verarbeiten, da sie weniger 
Rechenaufwand nach sich ziehen und weniger Speicherplatz 
benötigen. Außerdem ist die Programmierarbeit einfacher 
für RSSI-Werte.


\newpage
\subsection{Theorie}

Hierbei wurde sich unter anderem auf dieses 
\href{https://ieeexplore.ieee.org/stamp/% hier cut gesetzt
stamp.jsp?tp=&arnumber=6823677} % damit der Link keine
{Paper (Seite 1132)} %  zu große Zeilenlänge hat
und diese 
\href{https://limo.libis.be/primo-explore/%
fulldisplay?docid=LIRIAS1662210&context%
=L&vid=Lirias&search_scope=Lirias&tab%
=default_tab&lang=en_US&fromSitemap=1}
{Doktorarbeit (Seite 20 ff.)}
gestützt, welche kostenlos downloadbar sind.

\subsubsection*{Intradistanz}
\begin{comment}
VL 5, Folie 36:
Distance between responses for the same challenges. 
Shows measurements error


Doktorarbeit, S.20:
A PUF response intra-distance is a random variable
describing the distance between two PUF responses 
from the same PUF instance and using
the same challenge

Paper, S. 1132:
Intra-PUF variation: Defined as the number of bits
in a PUF response that vary when an identical
challenge is repeatedly queried on a given PUF
device in a changing environment. This variation is
due to this environmental change as well as 
statistical noise. As a result, it is commonly 
represented in the form of a statistical 
distribution. Intra-PUF variation is a measure of 
the reproducibility of responses from an individual 
PUF circuit.
\end{comment}

Beschreibt die Distanz zwischen den
\textit{responses} für die selbe \textit{challenge}. 
Man kann es als Indikator für die Größe des Messfehlers,
als auch laut dem Paper als Messwert für die 
Reproduzierbarkeit betrachten. Die Unterschiede entstehen 
durch Umwelteinflüsse und \textit{statistical noise}.\\

In der Praxis kann dieser Wert, ähnlich wie in 
den Kommentaren zu \verb|compute_intra_distance| 
von Aufgabe 1.3 beschrieben, berechnet werden: Die
erste Messung einer \textit{challenge} wird als Referenz 
genommen. Anschließend wird der euklidsche Abstand zu 
allen Messungen der selben \textit{challenge} berechnet.
Das wird für alle \textit{challenges} wiederholt.


\subsubsection*{Interdistanz}
\begin{comment}
VL 5, Folie 36:
Distance between responses for different challenges. 
Indicates uniqueness of responses.

Doktorarbeit, S. 22:
A PUF response inter-distance is a random variable 
describing the distance between two PUF responses 
from different PUF instances using the same challenge:


Paper, S. 1132:
Inter-PUF variation: Defined as the number of bits
in a PUF response that vary between different
devices for a set of shared challenges. This is 
due to differences between the physical ICs and 
is also commonly represented in the form of a 
statistical distribution. The inter-PUF variation 
is a measure of the uniqueness of an individual 
PUF circuit
\end{comment}

Distanz zwischen \textit{responses} für verschiedene 
\textit{challanges} und Indikator für die Einzigartigkeit 
der \textit{responses}.\\

In der Praxis, kann es ähnlich wie in der Aufgabe 1.3 
bei der Funktion \verb|compute_inter_distance| 
aussehen, wobei man hier die Anzahl der
zu berechnenden Distanzen gering halten möchte.
Deswegen beschränkt man sich auf die jeweils ersten 
Messungen jeder \textit{challange}: Dazu werden alle 
Distanzen aller Paare, ungeordnet und ohne Zurücklegen, 
gebildet von den ersten Messungen aller 
\textit{challanges}, was somit 
\footnotesize $\binom{40}{2}=780$ \normalsize
verschiedene Paare sind. 


\newpage
\subsection{Implementierung}

\begin{python}
def standardize_per_channel(data):
	print("standardize_per_channel...")
	data_copy = data.copy()
	new_data = []
	
	for measurement in data_copy: 
		#get channels
		channels=np.array(np.split(measurement, len(measurement)/114)) 
		# copied new data in 1D Array
		new_data.append([ ((chan-mean)/var).flatten() for chan,mean,var in zip(channels, channels.mean(axis=1), channels.var(axis=1)) ])
    
	print('sucessful.')
	return np.array(new_data)
\end{python}

\hfill \break
\hfill \break

\begin{python}

def euclidean_norm_per_channel(data):
	print("euclidean_norm_per_channel...")
	data_copy = data.copy()
	new_data = []
	
	# len of vector
	l = lambda vector_a: compute_euclidean_distance(vector_a, np.zeros_like(vector_a))
	
	for measurement in data_copy: 
		channels=np.array(np.split(measurement, len(measurement)/114))
		new_data.append([ (chan/l(chan)).flatten() for chan in channels ])


	print('sucessful.')
	return np.array(new_data)
\end{python}
\newpage
\begin{python}    
def compute_intra_distance(data, num_challenges):
	print("compute_intra_distance...")
	data_copy = data.copy()

	challenges = np.split(data_copy, num_challenges)
	intra_distances = [compute_euclidean_distance(challenge[0],measurement) 
                       for challenge in challenges
                       for measurement in challenge[1:]]
    
	print('sucessful.')
	return np.array(intra_distances)
\end{python}

\hfill \break
\hfill \break

\begin{python}
def compute_inter_distance(data, num_challenges):
	print("compute_inter_distance...")
	data_copy = data.copy()

	challenges = np.split(data_copy, num_challenges)
	inter_distances = [compute_euclidean_distance(chall[0],measurement)
                       for i,chall in enumerate(challenges)
                       for j,subchall in enumerate(challenges)
                       if subchall is not chall
                       for measurement in subchall[0 if i<j else 1:]]

	print('sucessful.')
	return np.array(inter_distances)
\end{python}
\newpage
\subsection{Analyse II}
\begin{itemize}
    \item Ist eine Überlappung zwischen Intra- und 
    Interdistanzen gut oder schlecht?
\end{itemize}

Laut Folie 36 der fünften Vorlesung sollte die 
Intradistanzen deutlich kleiner sein als die 
Interdistanzen, ansonsten hat man kein 
funktionierendes System: \glqq Measurement error 
must always be low enough 
to distinguish challenges (intra distance << inter 
distance). Otherwise, the system cannot be used!\grqq

\begin{itemize}
    \item Welche Bedeutung spielt die Größe der 
    Überlappung im Bezug auf die Realisierung einer
    strong-PUF?
\end{itemize}

Die Überlappung sollte gering, sein wenn man eine 
strong-PUF konstruieren möchte. Wie in der Vorlesung
erwähnt wurde zur Folie 36 kann man verschiedene
\textit{Responses} und \textit{Challenges} nicht 
außereinander halten, wenn die Intradistanz groß 
ist. Eine klare Einzigartigkeit der \textit{Responses}
ist erwünscht.


\begin{itemize}
    \item Treffen Sie zu den verschiedenen 
    Plots des Frameworks Aussagen, wie leicht 
    bzw. schwer es ein Angreifer hat, eine 
    Challenge richtig zu raten, wenn er nur 
    die Daten anderer Challenges besitzt. 
    Welche Rolle spielen dabei die 
    verschiedenen Arten des Preprocessings?
\end{itemize}


\verb|No Preprocessing|:
Überlappungs ist hier signifikant. Ein Angreifer hat hier 
mit Abstand die besten Möglichkeiten erfolgreich zu sein.\\


\verb|Euclidean Normalization per Channel|: Intradistanz 
ist gering und fällt streng monoton ab. Die Verteilung der
Interdistanzen hingegen ähnelt einem Hügel mit der Mitte um 
den Wert von etwa 1.8 herum und zu beiden Seiten streng 
monoton abfallend. Der Abstand zwischen den \textit{Peaks} 
beider  Verteilungen ist bisschen weniger als 1,8. Die 
Erfolgsaussichten eines Angreifers sind hier wesentlich 
geringer als oben.\\


\verb|Standardization per Channel|: Hier gibt ein wenig
Überschneidung. Dieses \textit{Preprocessing} bietet
im Vergleich zu den anderen \textit{Preprocessings}
einem Angreifer die besten Chancen.\\


\verb|Standardization per Frequency Bin|: Optisch ähnelt die 
Verteilung der euklidschen Normalisation pro Kanal.
Auffallend ist, dass die Verteilung bei Intradistanz nicht
streng monoton abfallend ist. Der Verteilung bei Interdistanz
erstreckt sich über fast eine Spannweite von 6. Hier hat 
der Angreifer die schlechtesten Chancen.

\begin{figure}[hbt!]
	\centering
		\includegraphics[width=1\textwidth]
		{Bilder/no_preprocessing.png}
		%\caption{no\_preprocessing.png}
		\label{fig:Label1.4.1}
\end{figure}



\begin{figure}[hbt!]
	\centering
		\includegraphics[width=1\textwidth]
		{Bilder/euclid_norm_channel.png}
		%\caption{euclid\_norm\_channel.png}
		\label{fig:Label1.4.2}
\end{figure}



\begin{figure}[hbt!]
	\centering
		\includegraphics[width=1\textwidth]
		{Bilder/std_channel.png}
		%\caption{std\_channel.png}
		\label{fig:Label1.4.3}
\end{figure}



\begin{figure}[hbt!]
	\centering
		\includegraphics[width=1\textwidth]
		{Bilder/std_freq_bin.png}
		%\caption{std\_freq\_bin.png}
		\label{fig:Label1.4.4}
\end{figure}

 \clearpage
\newpage
\section{Implementierung Quantisierer}


\subsection{Implementierung Jana Multibit}
\begin{python}
	def JanaMultibit(X):
		#(i)determine the Range of RSS measurements from the minimum and the maximum measured RSS values
		x_min = min(X)
		x_max = max(X)

		#(ii) find N, the number of bits that can be extracted per measurement
		# N is already implemented
		N = number_of_bits

		#(iii) divide the Range into m = 2^n equal sized intervals
		M = 2**N	# M intervals, len(Range) = M+1
		interval = abs((max(X)-min(X))/float(M)) # interval size
		
		Range =[] 	# intervals
		
		if interval == 0:	#min=max
			interval = 1

		while x_min <= x_max:	# fill in
			Range.append(x_min) 
			x_min += interval

		#(iv) choose an N bit assignment for each of the M intervals (for example use the Gray code sequence) 
		bit_assignment = utils.gray_code(N)

		#(v) for each RSS measurement, extract N bits depending on the interval in which the RSS measurement lies. 
		quantized=[]
		for x in X:
			for j in xrange(len(Range) - 1):
				if Range[j] <= x <= Range[j+1]: # including max and min values
					quantized.extend(bit_assignment[j])
					break
			if len(Range) == 1:	# min=max - only one value
				quantized.extend(bit_assignment[0])
		return quantized
\end{python}


\begin{comment}
% Beispiel für einen Hyperlink 	
\href{https://www.rub.de}{hier klicken}

% Beispiel für Bilder mit Caption und Referenz
\begin{figure}[hbt!]
	\centering
		\includegraphics[width=1\textwidth ]
		{Bilder/a3_rtl_sdr.jpg}
		\caption{Hier Caption einfügen}
		\label{fig:Labelx}
\end{figure}

% Als Beispiel wie man referenziert
~\ref{fig:Labelx}

% Beispiel für das Einfügen von Python Code
\begin{python}
print("Hello World")
\end{python}

% Beispiel für das Erzwingen von Abstand (selbe Zeile)
\hspace{0.5mm}

% Beispiel für das Erzwingen von Abstand (nach unten)
\\[0.7cm]

%Beispiel für einen Pseudocode
\subsection*{b) Pseudocode}
\begin{algorithm}
\caption{Pseudocode}
\begin{algorithmic}[1]
\State $range = max[RSS] - min[RSS]$
\State $N \in [0, log_2 RSS]$
\State $M = 2^N$
\State $RSS[] \to M$ intervalls $I[]$ of equal size
\State Choose $N$ bit assignment $\forall$ $M$ intervalls
\For{$t \to len(RSS[])$}
\For{$i \to len(I[]$}
\If{$RSS[t] \in I[i]$}
\State $bitstream \gets$ bit assignment
\EndIf
\EndFor
\EndFor\\
\Return $bitstream$
\end{algorithmic}
\end{algorithm}

% Beispiel für ein Tabelle mit vergrößerter Schrift
\Large
\begin{tabular}{ |p{3cm}|||p{3cm}|p{3cm}||p{3cm}|p{3cm}|}
    \hline
    \multicolumn{5}{|c|}{Teilaufgabe 1: $A\rightarrow B$} \\
    \hline
    Blockgröße & Mittel Bob & Median Bob & Mittel Eve & Median Eve\\
    \hline
    \hspace{3.2mm}30 & 0.2290 & 0.1756 & 0.1798 & 0.1428\\
    100 & 0.2219 & 0.1848 & 0.1341 & 0.1053\\
    200 & 0.2401 & 0.2377 & 0.0946 & 0.0748\\
    250 & 0.2591 & 0.2328 & 0.1495 & 0.1260\\
    300 & 0.2791 & 0.2108 & 0.1784 & 0.1130\\
    \hline
\end{tabular}
\normalsize

% Öffnende und schließende deutsche Anführungszeichen
\glqq Zitierter Text komm hierhin \grqq  

% Kaligraphische Buchstaben im mathematischen Stil (ua) für
% Mengen verwendet. Hier G 
\( \mathcal{G} \)

% Ein Ausdruck hier compute_intra_distance in 
Code-Style-Ausgabe bzw. Arial-font (?)
\verb|compute_intra_distance|
\end{comment}

\end{document}