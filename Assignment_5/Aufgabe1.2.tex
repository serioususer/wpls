\subsection{Theorie}

Hierbei wurde sich unter anderem auf dieses 
\href{https://ieeexplore.ieee.org/stamp/stamp.jsp?tp=&arnumber=6823677} 
{Paper (Seite 1132)} 
und diese 
\href{https://limo.libis.be/primo-explore/fulldisplay?docid=LIRIAS1662210&context%=L&vid=Lirias&search_scope=Lirias&tab=default_tab&lang=en_US&fromSitemap=1}
{Doktorarbeit (Seite 20 ff.)}
gestützt, welche kostenlos downloadbar sind.

\subsubsection*{Intradistanz}
\begin{comment}
VL 5, Folie 36:
Distance between responses for the same challenges. 
Shows measurements error


Doktorarbeit, S.20:
A PUF response intra-distance is a random variable
describing the distance between two PUF responses 
from the same PUF instance and using
the same challenge

Paper, S. 1132:
Intra-PUF variation: Defined as the number of bits
in a PUF response that vary when an identical
challenge is repeatedly queried on a given PUF
device in a changing environment. This variation is
due to this environmental change as well as 
statistical noise. As a result, it is commonly 
represented in the form of a statistical 
distribution. Intra-PUF variation is a measure of 
the reproducibility of responses from an individual 
PUF circuit.
\end{comment}

Beschreibt die Distanz zwischen den
\textit{Responses} für die selbe \textit{Challenge}. 
Man kann es als Indikator für die Größe des Messfehlers,
als auch laut dem Paper als Messwert für die 
Reproduzierbarkeit betrachten. Die Unterschiede entstehen 
durch Umwelteinflüsse und \textit{statistical Noise}.\\

In der Praxis kann dieser Wert, ähnlich wie in 
den Kommentaren zu \verb|compute_intra_distance| 
von Aufgabe 1.3 beschrieben, berechnet werden: Die
erste Messung einer \textit{Challenge} wird als Referenz 
genommen. Anschließend wird der euklidsche Abstand zu 
allen Messungen der selben \textit{Challenge} berechnet.
Das wird für alle \textit{Challenges} wiederholt.


\subsubsection*{Interdistanz}
\begin{comment}
VL 5, Folie 36:
Distance between responses for different challenges. 
Indicates uniqueness of responses.

Doktorarbeit, S. 22:
A PUF response inter-distance is a random variable 
describing the distance between two PUF responses 
from different PUF instances using the same challenge:


Paper, S. 1132:
Inter-PUF variation: Defined as the number of bits
in a PUF response that vary between different
devices for a set of shared challenges. This is 
due to differences between the physical ICs and 
is also commonly represented in the form of a 
statistical distribution. The inter-PUF variation 
is a measure of the uniqueness of an individual 
PUF circuit
\end{comment}

Distanz zwischen \textit{Responses} für verschiedene 
\textit{Challenges} und laut dem Paper ein Indikator 
für die Einzigartigkeit eines \textit{PUF circuit}
beziehungsweise laut Vorlesung 5, Folie 36 ein Indikator
für die Einzigartigkeit der \textit{responses}.\\ 


In der Praxis, kann es ähnlich wie in der Aufgabe 1.3 
bei der Funktion \verb|compute_inter_distance| 
aussehen, wobei man hier die Anzahl der
zu berechnenden Distanzen gering halten möchte.
Deswegen beschränkt man sich auf die jeweils ersten 
Messungen jeder \textit{Challenge}: Dazu werden alle 
Distanzen aller Paare, ungeordnet und ohne Zurücklegen, 
gebildet von den ersten Messungen aller 
\textit{Challenges}, was somit 
\footnotesize $\binom{40}{2}=780$ \normalsize
verschiedene Paare sind. 

