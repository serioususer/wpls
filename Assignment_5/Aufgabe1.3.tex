\subsection{Implementierung}

\begin{python}
def standardize_per_channel(data):
	print("standardize_per_channel...")
	data_copy = data.copy()
	new_data = []
	
	for measurement in data_copy: 
		#get channels
		channels=np.array(np.split(measurement, len(measurement)/114)) 
		# copied new data in 1D Array
		new_data.append([ ((chan-mean)/var).flatten() for chan,mean,var in zip(channels, channels.mean(axis=1), channels.var(axis=1)) ])
    
	print('successful.')
	return np.array(new_data)
\end{python}

\hfill \break
\hfill \break

\begin{python}

def euclidean_norm_per_channel(data):
	print("euclidean_norm_per_channel...")
	data_copy = data.copy()
	new_data = []
	
	# len of vector
	l = lambda vector_a: compute_euclidean_distance(vector_a, np.zeros_like(vector_a))
	
	for measurement in data_copy: 
		channels=np.array(np.split(measurement, len(measurement)/114))
		new_data.append([ (chan/l(chan)).flatten() for chan in channels ])


	print('successful.')
	return np.array(new_data)
\end{python}

\newpage

\begin{python}    
def compute_intra_distance(data, num_challenges):
	print("compute_intra_distance...")
	data_copy = data.copy()

	challenges = np.split(data_copy, num_challenges)
	intra_distances = [compute_euclidean_distance(challenge[0],measurement) 
                       for challenge in challenges
                       for measurement in challenge[1:]]
    
	print('successful.')
	return np.array(intra_distances)
\end{python}

\hfill \break
\hfill \break

\begin{python}
def compute_inter_distance(data, num_challenges):
	print("compute_inter_distance...")
	data_copy = data.copy()

	challenges = np.split(data_copy, num_challenges)
	inter_distances = [compute_euclidean_distance(chall[0],measurement)
                       for i,chall in enumerate(challenges)
                       for j,subchall in enumerate(challenges)
                       if subchall is not chall
                       for measurement in subchall[0 if i<j else 1:]]

	print('successful.')
	return np.array(inter_distances)
\end{python}